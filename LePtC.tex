\documentclass{leptc}
\usepackage{mathrsfs}
\begin{document}

\eq{\lim\limits_{x\to 1^+}\frac{(x^3-1)}{x-1}\varphi(x)=\lim\limits_{x\to 1^+}\frac{(x-1)(x^2+x+1)}{x-1}\varphi(x)=\lim\limits_{x\to 1^+}(x^2+x+1)\varphi(x)}

\enl{等价无穷小}\eq{\lim\limits_{x\to 0}\frac{\sin x}{x}=1}\com{记为\eq{\sin x \sim x}}

$\left.\frac{\partial x}{\partial y}\right|_{\text{wall}}$

\chap{2010真题答案}
一、1.\eq{2\pi}\quad \eq{f(k)=f(t)|_{t=kT_s}=f(t)|_{t=k\times1}=\sin t|_{t=k}=\sin k}\quad 由于\eq{f(k)=\sin k,\Omega_0=1\Rightarrow\frac{\Omega_0}{2\pi}=\frac{1}{2\pi}不是有理数,所以不是周期序列}

2.\eq{\int^{\infty}_{-\infty}|f(t)|<\infty}

3.\eq{[f(t)]^2=f(t)\cdot f(t)\leftrightarrow\frac{1}{2\pi}F(j\omega)*F(j\omega)}函数卷积定义域为两函数上下定义域之和,即\eq{2\omega_m}

4.\eq{|H(j\omega)|=\frac{\sqrt{1+\omega^2}}{\sqrt{1+\omega^2}}=1}\quad
\com{\eq{\phi(\omega)=\arctan\frac{\text{虚}}{\text{实}}}}
\eq{\phi(\omega)=\arctan\omega-[\arctan(-\omega)]=2\arctan\omega}\com{求相频,分子(虚部/实部)之和除以分母(虚部/实部)之和}\quad\eq{\because\phi(\omega)}不是\eq{\omega}的一个正比例函数,所以会产生失真

5.\eq{\delta(k)=U(k)-U(k-1)\Rightarrow h(k)=g(k)-g(k-1)=\left(\frac{1}{4}\right)^kU(k)-\left(\frac{1}{4}\right)^{k-1}U(k-1)}

二、1.A.数字信号=离散信号,模拟信号=连续信号

2.B.\eq{\delta(at)=\frac{1}{|a|}\delta(t)}

3.C.奇谐函数\com{\eq{f(t)=-f(t\pm\frac{T}{2})}}

4.\enl{线性}\eq{y(k)=a_1y_1(k)+a_2y_2(k)},\eq{f(k)=a_1f_1(k)+a_2f_2(k)},
左边\eq{=a_1y_1(k)+a_2y_2(k)+[a_1y_1(k-1)+a_2y_2(k-1)]
\cdot[a_1y_1(k-2)+a_2y_2(k-2)]=a_1y_1(k)+a_2y_2(k)
+a^2_1y_1(k-1)y_1(k-2)+a^2_2y_1(k-1)y_2(k-2)+a_1a_2
[y_1(k-1)y_2(k-2)+y_1(k-2)y_2(k-1)]}

\includegraphics[width=.6\textwidth]{b.png}

\includegraphics[width=.6\textwidth]{c.png}

两周期最小公倍数\includegraphics[width=.6\textwidth]{d.png}
\chap{极限}
$\mathcal{N}$
\begin{equation*}
\left\{\begin{aligned}
B'&=-\partial\times E,dsssssssssssss\\
  &+df\times fdslfja\\
E'&=\partial\times B - 4\pi j,
\end{aligned}\right.
\quad \text{手动编号a-5}
\end{equation*}

\begin{equation*}
\left\{\begin{aligned}
x_{1}(t) &= e^{-\int_{t_{0}}^{t}
[k-\rho+\frac{\rho}{M}x^{*}+\alpha f(x_{3}(s))]ds}x_{1}(t_{0})\\
  &+  \int_{t_{0}}^{t}  e^{-\int_{v}^{t}
 [k-\rho+\frac{\rho}{M}x^{*}+\alpha
f(x_{3}(s))]ds}
[-\frac{\rho}{M}x(v)x_{1}(v)-\alpha x^{*}f(x_{3}(v))]\}dv\\
x_{2}(t) &= e^{-\int_{t_{0}}^{t} [\gamma+\beta
h(x_{4}(s))]ds}x_{2}(t_{0})
  +  \int_{t_{0}}^{t}  e^{-\int_{v}^{t}[\gamma+\beta
h(x_{4}(s))]ds} \{\alpha e^{-m\tau}x(v-\tau)f(x_{3}(v-\tau))\}dv\\
x_{3}(t)
 &=e^{-d(t-t_{0})}x_{3}(t_{0})+\int_{t_{0}}^{t}e^{-d(t-v)}px_{2}(v)dv\\
x_{4}(t)
 &=e^{-q(t-t_{0})}x_{4}(t_{0})+\int_{t_{0}}^{t}e^{-q(t-v)}\delta
 x_{2}(v)dv
\end{aligned}\right.
\quad \text{手动编号a-5}
\end{equation*}


$$
 \left\{\begin{array}{ll}
x_{1}(t) =e^{-\int_{t_{0}}^{t} [k-\rho+\frac{\rho}{M}x^{*}+\alpha
f(x_{3}(s))]ds}x_{1}(t_{0})
  +  \int_{t_{0}}^{t}  e^{-\int_{v}^{t}
 [k-\rho+\frac{\rho}{M}x^{*}+\alpha
f(x_{3}(s))]ds}
[-\frac{\rho}{M}x(v)x_{1}(v)-\alpha x^{*}f(x_{3}(v))]\}dv\\

x_{2}(t) = e^{-\int_{t_{0}}^{t} [\gamma+\beta
h(x_{4}(s))]ds}x_{2}(t_{0})
  +  \int_{t_{0}}^{t}  e^{-\int_{v}^{t}[\gamma+\beta
h(x_{4}(s))]ds} \{\alpha e^{-m\tau}x(v-\tau)f(x_{3}(v-\tau))\}dv\\

 x_{3}(t)
 =e^{-d(t-t_{0})}x_{3}(t_{0})+\int_{t_{0}}^{t}e^{-d(t-v)}px_{2}(v)dv\\

x_{4}(t)
 =e^{-q(t-t_{0})}x_{4}(t_{0})+\int_{t_{0}}^{t}e^{-q(t-v)}\delta
 x_{2}(v)dv
\end{array}\right.\eqno (2.8)$$

求此方程特解:\eq{y''-2y'+2y=x\e^x\cos x}

原函数\eq{F(x)=\ln^2x},\eq{f(x)=F'(x)=2\ln x\cdot \frac{1}{x},\int xf'(x)\dif x=\int x\dif f(x)=xf(x)-\int f(x)\dif x=xf(x)+F(x)=\\2\ln x+\ln^2x}

\ent[\B Analysis]{分析}\quad 变量替换+洛比达

\eqd{\lim_{x\to +\infty}x\left(a^{\frac{1}{x}}-b^{\frac{1}{x}}\right)=
\lim_{x\to +\infty}\frac{\left(a^{\frac{1}{x}}-b^{\frac{1}{x}}\right)}{\frac{1}{x}}
\xlongequal{\text{令}t=\frac{1}{x}}\lim_{t\to 0^+}\frac{a^t-b^t}{t}\xlongequal{\text{洛比达}}\lim_{t\to 0^+}a^t\ln a-b^t\ln b=\ln\frac{a}{b}}


(一)“\eq{1^\infty}”,\prv{设\eq{\lim f(x)=0},\eq{\lim g(x)=\infty},则
\eq{\lim[1+f(x)]^{g(x)}\com{(1^\infty)}\xlongequal{N=\e^{\ln N}}\e^{\lim g(x)\ln[1+f(x)]}\xlongequal{\com{\ln(1+x)\sim x}}\e^{\lim f(x)g(x)}}}

\ent[\B Remember]{记住}~\eq{1^\infty=\e^A},\eq{A}是括号中\eq{1}后的部分,底数\eq{f(x)}与指数幂\eq{g(x)}乘积的极限.

\enl{例}~设\eq{f''(x)}连续,且\eq{\lim\limits_{x\to0}\left(1+x+\frac{f(x)}{x}\right)^{\frac{1}{x}}=\e^3}. 求\eq{f(0),f'(0),f''(0)}及\eq{\lim\limits_{x\to0}\left(x+\frac{f(x)}{x}\right)\cdot\frac{1}{x}}.

\enl{例}~求\eq{\lim\limits_{x\to0}\frac{\e^x-\e^{\sin x}}{x-\sin x}}

\enl{解}~\eq{I=\lim\limits_{x\to0}\frac{\e^{\sin x}\left(\e^{x-\sin x}-1\right)}{x-\sin x}\xlongequal{\com{\e^x-1\sim x}}\lim\limits_{x\to0}\frac{\e^{\sin x}\left(x-\sin x\right)}{x-\sin x}=1}
\link[笔记名]{章节名}

\chap{不等式证明}

\ent[\B Remember]{记住}~区间内不等式的证明,首先应想到利用函数的单调增减性来证明.

\enl{证明}~\eq{x\ln\frac{1+x}{1-x}+\cos x\geqslant 1-\frac{x^2}{2}},(-1<x<1)

\enl{例}~设\eq{f'(x)}在\eq{[0,1]}上连续,\eq{f(1)-f(0)=1},证明\eq{\int^{1}_{0}f'(x)\dif x\geqslant1}

证:\com{函数平方的积分应该这样证}\eq{\underline{\bigl[f'(x)-1\bigr]^2\geqslant0}}\Rightarrow\eq{f'^2(x)-f'(x)+1\geqslant0}\Rightarrow
\eq{\int^1_0\bigl[f'^2(x)\bigr]\dif x\geqslant2\int^1_0f'(x)\dif x-\int^1_01\dif x=2f(x)\big|^1_0-1=2\bigl[f(1)-f(0)\bigr]-1=2-1=1.\com{(\because f(1)-f(0)=1)}}

\chap{一元微积分的应用}
\ent[\B Take \B Care]{注意}
知道\eq{f(x)}在\eq{(a,b)}内可导,又知\eq{f(a)=0}\com{或\eq{f(b)=0}}的命题,通常要利用拉格朗日中值定理将\eq{f(x)}
写成\eq{f(x)=f(x)-f(a)=\xi f(\xi)}\com{或\eq{f(x)=f(x)-f(b)=\xi f(\xi)}},其中\eq{\xi\in(x,a)}\com{或\eq{\xi\in(x,a)}}.

\ent[\B Take \B Care]{注意}
\eq{\frac{f(x)-f(x_0)}{x-x_0}=k+\alpha(x),\text{其中}\lim\limits_{x\to x_0}\alpha=0},当\eq{x}在\eq{x_0}的充分小邻域内时,\eq{(k+\alpha(x))}与\eq{k}同号.

\chap{双语彩色笔记模版}

作者:\href{mailto:alileptc@gmail.com}{\LePtC}

项目主页:\url{https://github.com/LePtC/LeNote }

笔记主页:\url{http://leptc.github.io/lenote }

使用 \href{http://opensource.org/licenses/MIT}{MIT 开源协议}

\compiled


\chap{安装}

\ent[install TeX]{安装\TeX 系统}
Windows 系统可选择安装
\href{http://miktex.org/download}{MiKTeX}
然后选择自动安装缺失的包,或直接安装
\href{http://www.ctex.org/CTeXDownload }{CTeX Full}
或 \href{http://www.ctan.org/tex-archive/systems/texlive/Images/ }{TeXLive iso} ,
前两者是把 \code{leptc.cls} 放到
\code{CTeX/MiKTeX/tex/latex/} 目录下,
然后在 MiKTeX 的 Settings 里面点 Refresh FNDB 即可,
后者是在 \code{texlive/2014/texmf.cnf} 末尾加上
\\ \code{TEXMFLOCAL = $SELFAUTOPARENT/../texmf-local,E:/blabla/(anypath)},
\\然后把\code{leptc.cls} 放到
\code{(anypath)/tex/latex/misc} 这个路径中,
在命令行执行 \code{texhash} 即可

\ent[compiler]{编译器}
只有 latex+dvipdfmx 或 xelatex 编译出的 pdf 能正确复制,
前者请参考文件 \code{leptc.sty}

dvipdfmx 方案本狸已停止更新,推荐使用 %\XeTeX 方案,
xelatex 的编译命令及常用选项:

\code{xelatex --quiet --synctex=1 -interaction=nonstopmode $(NAME_PART).tex}

xelatex 需要多编译几遍才能正确生成书签,
详见 \href{https://github.com/LePtC/LeNote}{项目主页}的
\code{compile} 文件夹

\com{xelatex.exe 等编译器均在
\code{CTeX/MiKTeX/miktex/bin/}
或 \code{texlive/2014/bin/win32} 目录下,
如果命令行没有此命令,可在命令中输入 exe 的完整路径,
或手动将路径添加到系统的环境变量并重启}

\ent[editor]{编辑器}
\href{http://tex.stackexchange.com/questions/339/latex-editors-ides }{各种编辑器的比较},
有关编辑器不同的设置方法见\href{https://github.com/LePtC/LeNote}{项目主页}的 \code{README.md}

\ent[reader]{阅读器} 推荐使用
\href{http://blog.kowalczyk.info/software/sumatrapdf/download-free-pdf-viewer-cn.html }{SumatraPDF}
来查看 pdf, 有
\href{http://xhmikosr.1f0.de/sumatrapdf/ }{64 位版本}
\com{非官方}

支持 synctex,需在 \code{InverseSearchCmdLine} 里填入相应编辑器的反向查找命令

Notepad++:
\code{\"C:\\Program\ Files\ (x86)\\Notepad++\\notepad++.exe\" -n\%l \"\%f\"}

Sublime:
\code{\"C:\\Program\ Files\\Sublime\\sublime_text.exe\" \"\%f:\%l\"}

\ent[tex file]{\TeX 文档}
新建filename.tex,存为 UTF-8 无 BOM 格式,
开头为 \verb|\documentclass{leptc}|,
然后就可以在 \verb|\begin{document} ... \end{document}| 之间写正文啦,
喵 \tld

\com{待解决:文档名不能有空格否则不能识别,
不能有中文否则会报错}




\chap{章节}

\begin{tabular}{lcll}

	章节
	&\com{效果见右上方\eq{\nearrow} }
	&\verb|\chap{中文}|
	&\multicolumn{1}{c}{文本}\\

  双语词条
	&\ent[\B Superconducting \B{QU}antum \B Interference \B Device]{超导量子干涉器}
	&\verb|\ent[\B Entry]{词条} |
	&居中用 \verb|\entc| 		\\

  双语正文
	&\eng[English translation]{注英文}
	&\verb|\eng[English]{正文} |
	& 用 \verb|\engr| 则英文标在右侧 		\\

  标签
	&\enl{标签}
	&\verb|\enl{标签} |
	& 用于\enl{例},\enl{定理},\enl{推论}等		\\

	inline公式
	&\eq{f(x,y)=\frac{\e^x}{y}}
	&\verb|\eq{\frac{\e^x}{y}}|
	&长公式不用 \verb|$$|, 括号便于配对	\\

	display公式
	&\eqd{f(x,y)=\frac{\e^x}{y}}
	&\verb|\eqd{\frac{\e^x}{y}}|
	&修改公式模式只需加一个 \verb|d|即可	\\

  圆括号表注释
	&\com{注释}
	&\verb|\com{注释}|
	&多行注释: \verb|\coms{注\\释}|	\\

	方括号表证明
	&\eq{\vec{v}=\prv{\od{}{t}(r\ve{r})=}\dot{r}\ve{r}+r\dot\theta\ve{\theta}}
	&\verb|\prv{blabla=}|
	&灰色的优先级低于自动高亮 	\\

	尖括号表链接
	&\link{颜色}
	&\verb|\link[笔记名]{章节名}|
	&同一笔记内的链接笔记名可省略	\\

	贴图
	&\figin[0.05]{ali}
	&\verb|\fig[相对宽度]{图片名}|
	&内置: \verb|\figin| 多图并排: \verb|\figgg|	\\

\end{tabular}


\chap{实例}


\com{本笔记均指实数域} \ent[orthogonal group]{正交群}
\eq{\mathbf{O}(n)}
需 \eq{\frac12n(n-1)} 个独立参数
\prv{约束方程\eq{O^TO=I}上下三角的$=0$对称}


\eq{\mathbf{O}(n)=\mathbf{SO}(n)\otimes\{I,-I\}}
\prveq{\abs{O}=\pm1}
\enl{例}
\eq{\mathbf{O}(1)=\{\pm1\},\ \mathbf{SO}(1)=\{1\}}

\ent{二维空间转动群}
\eq{\mathbf{SO}(2)=\{R_z(\theta)|-\pi\le\theta\le\pi\}}
\enl{例}
\eq{\mathbf{D}_n} 是 \eq{\mathbf{O}(2)} 的离散子群
\com{反射对应行列式 $-1$}

\com{参数群可用数学分析方法}
\prvs{
由于\eq{\mathbf{SO}(2)}阿贝尔,表示一维,设 \eq{A=\{a(\theta)\}},
已知乘法关系为 \eq{a(\theta_1+\theta_2)=a(\theta_1)a(\theta_2)},
两边对\eq{\theta_1}求导后令\eq{\theta_1=0},
得\eq{a'(\theta_2)=a(\theta_2)a'(0)},
为使幺正取\eq{a'(0)=\ii m}纯虚,解得\eq{a(\theta)=\e^{\ii m\theta}},
由周期性\eq{a(\theta)=a(\theta+2\pi)}\com{费米子是\eq{+4\pi}},
得\eq{m\in\mathbb{Z}},然后证完备
}

\ent[three dimensional rotation group]{三维空间转动群}
\eq{\mathbf{SO}(3)\nors\mathbf{O}(3)},
均由3个\ent{群参数}表示 \com{独立,实数}, 群元素写法:

\N1 \eq{R_{(\theta,\varphi)}(\psi),\ 0\le\psi\le\pi}
\to 映射到半径 \eq{\pi} 球面上 \eq{(\psi,\theta,\varphi)}
\com{球面上的点二对一 \eq{R_n(\pi)=R_{-n}(\pi)}} \link{拓扑}





\chap{图片混排}

图片混排的命令为 \verb|\figr{ali.jpg}{0.1}{很多行文字}|, 实例 \eq{\downarrow}
\ \\

\figr{natural.png}{0.22}
{
\ent[arc length]{弧长} \eq{s=s(t),\ \vec{r}=\vec{r}(s)}
\com{可任意选定 \eq{s} 的零点和正向,与运动方向无关}

\ent[tangential]{切向} \eq{\ve{t}=\frac{\dif \vec{r}}{\dif s}},
\eq{\od{}{ \theta}\ve{t}=\ve{n}\ \to}
\ent[normal]{法向}指向曲线凹侧, \eq{\od{}{\theta}\ve{n}=-\ve{t}},
\eq{\ved{t}=\od{\ve{t}}{\theta}\od{\theta}{s}\dot s=\ve{n}\frac{1}{\rho}v}

\eq{\vec{v}=\dot s\ve{t}},
\eq{\vec{a}=\ddot s\ve{t}+\frac{v^2}{\rho}\ve{n}},
\ent[curvature radius]{曲率半径} \eq{\rho=\od{s}{\theta}=(1+y'^2)^{\frac{3}{2}} / \abs{y''}},
常用 \eq{a_t=\dot v=\od{v}{s}v}

加速度既反映速度大小也反映方向变化
\eq{a_t=\od{v}{t},\ a_n=\frac{v^2}{\rho},\
a=\sqrt{a_t^2+a_n^2},\ \tan\theta=\frac{a_n}{a_t}}
}


\chap{表格混排}

表格混排的命令为 \verb|\tabr[0.4]{很多行文字}{很多行表格}|, 实例 \eq{\downarrow}
\ \\


\tabr[0.72]{
\enl{性质} 同类元素的特征标相等 \com{记类中元素个数为 \eq{n_i}, 求和公式中可合并}

群的$\forall\ne$IUR的个数等于群中类的个数 \eq{r} \to 特征标表是方阵

\ent{第一正交性关系} 特征标表各行正交
\eq{\frac{1}{n}\sum^r n_i \chi^{(p)*}(g) \chi^{(q)}(g)=\delfun_{pq}}

\ent{第二正交性关系} 特征标表各列正交
\eq{\frac{n_i}{n}\sum^r_p \chi^{(p)*}(g_i) \chi^{(p)}(g_{i\co})=\delfun_{ii\co}}

}{

\begin{tabular}{|c|c|c|c|}
\hline
  特征标 &\eq{e} &\eq{r_1,r_2} &\eq{a,b,c} \\
\hline
  \eq{\chi^S} &1 &1 &1 \\
  \eq{\chi^A} &1 &1 &\eq{-1} \\
  \eq{\chi^\Gamma} &2 &\eq{-1} &0 \\
\hline
\end{tabular}
}


\chap{颜色}

模版对以下情况做自动高亮:
\prv{更新:绿色为注释专用, 算符改用橙色, 章节由红色改为紫色}
\ \\



\hspace{-10pt}
\begin{tabular}{lccl}

  推导为绿色
  &\eq{\to \ns \Rightarrow}
  &\verb|\to \ns \Rightarrow|
  & \\

	函数名橙色
	&\eq{\sin(x+y),\exp[x+y]}
	&\verb|\e^{x+y},\exp[x+y]|
	&自然对数 \eq{\e^x} 变色,命令为 \verb|\e| \\

	算符\sout{绿色}
	&\eq{\dif x,\Dif x,\delta x,\Delta x,\nabla x}
	&\verb|\dif x,\delta x,\nabla x|
	&默认高亮,不高亮用 \verb|\olddelta| \\

	物理单位蓝色
	&\eq{\oC,6.67\E{-11}\uni{m^3/(kg\cdot s^2)}}
	&\verb|\uni{m^3/(kg\cdot s^2)}|
	&虚数单位 \eq{\ii} 变色,命令为 \verb|\ii| \\

\end{tabular}



\chap{字体}

正文默认字体: Adobe 仿宋,\textbf{词条 Adobe 黑体},
英文 Times New Roman,\engr[Verdana]{英文翻译}

\prv{2015.05 更新:为改善斜杠的显示 \ent{例/例}, 黑体字体改为方正准圆}
\ \\

为了避免命名空间冲突,为了世界的和平,强迫症如下规定数学字体的含义:
\ \\

打字机体 \verb|\texttt{}|用于源代码: \code{file.tex}
\ \\

\begin{tabular}{lcl}

	所有变量、粒子符号为斜体
	&\eq{x,y,z,r,v,a,e,n,p}
	&\com{公式环境下默认为斜体} \\

	其它字母、元素符号为正体
	&\eq{\Ek,\kB,\NA,F\inter,\cc,\ce{He}}
	&\verb|\mathrm{}| \\

	双线体注册为数域
	&\eq{\mathbb{N,Z,Q,A,R,C,H}}
	&\verb|\mathbb{}| \\

	花体注册为泛函 %和大O记号?
	&\eq{\mathcal{L,F,Z}}
	&\verb|\mathcal{}| \\

  粗体注册为群
  &\eq{\mathbf{D}_n,\mathbf{U}(n),\mathbf{SO}(3)}
  &\verb|\mathbf{}| \\

  哥特体注册为代数
  &\eq{\mathfrak{su}(n),\mathfrak{so}(3)}
  &\verb|\mathfrak{}| \\

  特殊符号
  &电动势 \eq{\emf}
  &\verb|\emf| 使用 \verb|\mathscr{}| \\

\end{tabular}



\chap{其它符号范例}

\begin{tabular}{lcl}

  大圈小圈
  &\N1 \N2 \n1 \n2
  &\verb|\N1 \N2 \n1 \n2|\\

  区分求导/撇
  &\eq{y',y\co,y\co[x]}
  &\verb|y',y\co,y\co[x]|\\

	矢量
	&\eq{\vec{OA},\vec{p_c}',\vecd{p},\ve{r}}
	&\verb|\vec{OA},\vec{p_c}',\vecd{p},\ve{r}|\\

	张量
	&\eq{\vvecd{T},\vvvec{\varepsilon}}
	&\verb|\vvecd{T},\vvvec{\varepsilon}|\\

	矢量算符
	&\eq{\hatv{p},\hatvs{S}}
	&\verb|\hatv{p},\hatvs{S}|\\

  矢量微分
  &\eq{\nabla x,\nablad \vec x,\nablat \vec x,\nablas x}
  &\verb|\nabla x,\nablad \vec x,\nablat \vec x,\nablas x|\\

\vspace{3pt}\hspace{-4pt}
  导数,偏导数
  &\eqd{\od{y}{x},\pd[2]{L}{x},\md{L}{4}{x}{2}{y}{2}}
  &\verb|\od{y}{x},\pd[2]{L}{x},\md{L}{4}{x}{2}{y}{2}|\\

	某处的导数
	&\eq{\odat{y}{x}{x_0},}
	\eqd{\odat{y}{x}{x_0},\pdat{L}{x}{y,z}}
	&\verb|\odat{y}{x}{x_0},\pdat{L}{x}{y,z}|\\

\vspace{3pt}\hspace{-4pt}
	圈积分
	&\eqd{\oiint_S \vec{B} \cdot\dif \vec{S}= \oint_L \vec{A} \cdot\dif \vec{l}}
	&\verb|\oiint_S \oint_L|\\

	推导上加字
	&\eq{\xlongequal{\text{归一}}, \xrightarrow{\times a^2}}
	&\verb|\xlongequal{\text{}} \xrightarrow{}|\\

  左花括号
  &\eq{\delta _{ij} = \leftB[2]{\matn{1 &(i = j)\\ 0 &(i \ne j)}}}
  &\verb|\leftB[行数]{\matn{1 &(i = j)\\ 0 &(i \ne j)}}|\\

  矩阵,行列式
  &\eq{\mat[0.8]{1&0\\0&1},\matd[0.8]{-a&b\\c&-d}}
  &\verb|\mat{1&0\\0&1},\matd{-a&b\\c&-d}|\\

  杨图,杨盘
  &\ynd[0.5]{3,1}\quad$T^{[21]}_1=$\yng{1&2\\3}
  &\verb|\ynd{3,1},\yng{1&2\\3}|\\

\end{tabular}

\ \\
太多了 ... 慢慢写



\chap{学习网站}

\url{http://tex.stackexchange.com/ }

\href{http://linux-wiki.cn/wiki/zh-hans/LaTeX%E4%B8%AD%E6%96%87%E6%8E%92%E7%89%88%EF%BC%88%E4%BD%BF%E7%94%A8XeTeX%EF%BC%89 }{ LaTeX中文排版(使用XeTeX)}

\href{http://www.wikibooks.org }{维基 book}






\end{document}
